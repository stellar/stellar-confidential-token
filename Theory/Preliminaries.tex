\section{Preliminaries}\label{sec:Preliminaries}

\subsection{Notation} 
We use calligraphic letters to denote sets, uppercase for random variables, and lowercase for integers and functions. %All logarithms considered here are base $2$. 
Let $\N$ denote the set of natural numbers.  For $n\in \N$, let $[n] \eqdef \set{1,\ldots,n}$ and $(n) \eqdef \set{0,\ldots,n}$.  For a relation $\cR$, let $\cL(\cR)$ denote its underlying language, \ie  $\cL(\cR) \eqdef \set{x\colon \exists w \colon (x,w) \in \cR}$.    We number the element of a vector starting from $0$, \ie $v_0,v_1,
\ldots,$


\subsection{Homomorphic Encryption}

An homomorphic encryption over $\Z_q$ is a triplet $(\KeyGen,\Enc,\Dec)$ of efficient algorithms, with the standard correctness and semantic security properties.  In addition, there exist an efficient  addition operation denote $+$  such that for uniformly  generated public key  $\pk$, and any two  messages $x_0,x_1\in \Z_q$,  it holds that  $\Enc_\pk(x_0) + \Enc_\pk(x_1) $ are computationally indistinguishable from $\Enc_\pk(x_0+ x_1 \bmod q)$.


\subsection{Security Model}

\Inote{UC}


\Inote{sid}
